\documentclass[letterpaper]{article}
\usepackage{amsmath}
\usepackage{amsfonts} % for \mathbb{R}
\usepackage{alg}
\usepackage[margin=1.25in]{geometry}


\begin{document}
\thispagestyle{empty}
\pagestyle{empty} % no page numbers

\author{Dylan Hutchison}
\title{A Comparison between Dempster-Shafer and Bayesian Approaches to Soft Evidence}
\date{last updated \today}
%\institute{Stevens Institute of Technology\\
%Hoboken, NJ 07030
%\email{dhutchis@stevens.edu}}
\maketitle

%\noindent Dylan Hutchison\\
%CS 810B Causal Inference\\
%Project Proposal due 29 October 2012
%\begin{center}
%A Comparison between Dempster-Shafer and Bayesian Approaches to Soft Evidence
%\end{center}

\section{3-State Single Variable}
Let's consider the case of a murder investigation that after considerable detective work has only three possible suspects: Peter, Paul and Mary.(CITE the Shafer book)  Let G be the random variable denoting the murderer and let the domain of G be \{Peter, Paul, Mary\}.  Let's see how the Bayesian and Dempster-Shafer approaches handle this scenario with varying degrees of evidence. \cite{Shafer1976}

As single variable offers little chance of ambiguity, I will use the shorthand P(Peter) to denote P(G=Peter) frequently.

\subsection{No evidence}
Bayesian theory tells us to impart equal prior probability to the possible states of G if there is no other information.  Thus P(Peter)=P(Paul)=P(Mary)= $\tfrac{1}{3}$.

Dempster-Shafer theory will impart no belief to any of the three single states of G.

\bibliographystyle{splncs}
\bibliography{BayesDSComparison}

\end{document}